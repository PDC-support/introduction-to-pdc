\section*{Login and running}

\begin{frame}[fragile]
  \frametitle{Hands-on exercise}

\footnotesize
\begin{exampleblock}{\large{Login}}
\begin{itemize}
  \item Some configuration steps are needed to log in to PDC 
  \item Depends on OS: https://www.pdc.kth.se/support/documents/login/login.html
  \item In short, Kerberos and SSH supporting GSSAPI key exchange must be installed 
  \item If needed, you will receive help to connect from your own laptops
\end{itemize}
\end{exampleblock}

\begin{exampleblock}{\large{Exercises}}
\begin{itemize}
  \item You will now practice key steps in using PDC resources
  \item The example source code and batch scripts can be found at \verb|/afs/pdc.kth.se/home/k/kthw/Public/pdc_test.tar.gz|,
    or in the GitHub repository \verb|https://github.com/PDC-support/introduction-to-pdc| (in the intro-course branch)
\end{itemize}
\end{exampleblock}

\end{frame}


%-------------------------------------------------------------------------------------
\begin{frame}[fragile]
  \frametitle{Login}
\begin{itemize}
  \item Start by logging in to Beskow
  \item Where are you after login?
    \begin{itemize}
      \item What is your current directory?
      \item What's the name of the login node?
    \end{itemize}
  \item Have a look at the currently running processes on the login node
\end{itemize}
\end{frame}


%-------------------------------------------------------------------------------------
\begin{frame}[fragile]
  \frametitle{Login}
\begin{exampleblock}{{Answer}}
    \verbatimfont{\footnotesize}
    \begin{itemize}
    \item The \verb|pwd| command shows your current directory:
    \begin{verbatim}
      $ pwd
      /afs/pdc.kth.se/home/k/kthw
    \end{verbatim}

    \item The \verb|hostname| command shows the hostname of the login node:
    \begin{verbatim}
      $ hostname
      beskow-login2.pdc.kth.se
    \end{verbatim}

    \item The \verb|top| command shows a snapshot of currently running processes:
    \begin{verbatim}
      $ top
    \end{verbatim}

    \item Note that the login node is a shared resource!
    \end{itemize}

\end{exampleblock}
\end{frame}


%-------------------------------------------------------------------------------------
\begin{frame}[fragile]
  \frametitle{Disk quota, listing projects (allocations)}
\begin{itemize}
  \item Check your AFS disk quota (hint: you will need the \verb|fs| command, type \verb|fs help| to see available subcommands)
  \item Go to your klemming nobackup directory. Can you run \verb|fs| there?
  \item Check which allocation(s) you belong to using the \verb|projinfo| command
    \begin{itemize}
      \item Check all options of \verb|projinfo| using the \verb|-h| flag
      \item How much have your allocations been used, and for how long are they valid?
    \end{itemize}
\end{itemize}
\end{frame}


%-------------------------------------------------------------------------------------
\begin{frame}[fragile]
  \frametitle{Disk quota, listing projects (allocations)}
\begin{exampleblock}{{Answer}}
    \verbatimfont{\footnotesize}
    \begin{itemize}
    \item The \verb|fs lq| command shows the name of the AFS volume of your home directory, your disk quota and usage:
    \begin{verbatim}
      $ fs lq
    \end{verbatim}

    \item \verb|fs lq| only works on AFS. You can type \verb|df -h .| instead, but there's no quota on your klemming usage:
    \begin{verbatim}
      $ df -h .
Filesystem     Size  Used Avail Use% Mounted on
...:/klemming  5.2P  4.4P  730T  86% /cfs/klemming
    \end{verbatim}

    \item The \verb|projinfo| command accesses a database that contains logs of all allocations 
    \begin{verbatim}
      $ projinfo
    \end{verbatim}

    \end{itemize}

\end{exampleblock}
\end{frame}




%-------------------------------------------------------------------------------------
\begin{frame}[fragile]
  \frametitle{Working with modules}
\begin{itemize}
  \item List the modules that are currently loaded (hint: type \verb|module help|)
  \item List all the modules that are available on Beskow
  \item List all available modules named \verb|PrgEnv|
  \item Swap from the \verb|PrgEnv-cray| to the \verb|PrgEnv-gnu| module
\end{itemize}
\end{frame}


%-------------------------------------------------------------------------------------
\begin{frame}[fragile]
  \frametitle{Working with  modules}
\begin{exampleblock}{{Answer}}
    \verbatimfont{\footnotesize}
    \begin{itemize}
    \item Listing all loaded modules and all available modules is done like this:
    \begin{verbatim}
      $ module list
      $ module avail
    \end{verbatim}

    \item To list all modules matching a pattern (like PrgEnv), type
    \begin{verbatim}
      $ module avail PrgEnv
    \end{verbatim}

    \item To swap programming environments (i.e. compiler environments), type
    \begin{verbatim}
      $ module swap PrgEnv-cray PrgEnv-gnu
    \end{verbatim}
    After swapping these modules, the compiler wrappers \verb|CC, cc| and \verb|ftn| point to the GNU compilers (instead of the 
    Cray compilers)

    \end{itemize}

\end{exampleblock}
\end{frame}

%-------------------------------------------------------------------------------------
\begin{frame}[fragile]
  \frametitle{Compiling code}
\begin{itemize}
  \item Copy the tarball \verb|/afs/pdc.kth.se/home/k/kthw/Public/pdc_test.tar.gz| to  
    your nobackup directory, and unpack it
  \item Now compile the MPI example code \verb|hello_world_mpi.c|
    \begin{itemize}
      \item Do you need to load any MPI libraries?
      \item What compiler will be used when using the \verb|cc| compiler wrapper?
    \end{itemize}
\end{itemize}
\end{frame}


%-------------------------------------------------------------------------------------
\begin{frame}[fragile]
  \frametitle{Compiling code}
\begin{exampleblock}{{Answer}}
    \verbatimfont{\footnotesize}
    \begin{itemize}
    \item Copying and extracting the tarball to your klemming directory:
    \begin{verbatim}
      $ cd /cfs/klemming/nobackup/k/kthw
      $ cp /afs/pdc.kth.se/home/k/kthw/Public/pdc_test.tar.gz .
      $ tar zxf pdc_test.tar.gz
      $ cd pdc_test
    \end{verbatim}

    \item When using the compiler wrappers CC, cc and ftn, no MPI or numerical libraries need to be loaded 
      or linked to explicitly!
    \begin{verbatim}
      $ cc -o hello_world_mpi hello_world_mpi.c
    \end{verbatim}
    \item Which compiler did we just compile with?
    \begin{verbatim}
      $ cc --version
      gcc (GCC) 4.9.1 20140716 (Cray Inc.)
    \end{verbatim}

    \end{itemize}

\end{exampleblock}
\end{frame}


%-------------------------------------------------------------------------------------
\begin{frame}[fragile]
  \frametitle{Submitting jobs}
\begin{itemize}
  \item Open the batch script \verb|sbatch_beskow.sh| in your favorite editor (emacs or vim)
    \begin{itemize}
      \item Set the allocation ID to \verb|edu18.intropdc|
      \item Set that the job should run on two nodes, with 16 processes running on each node
      \item Set the requested time of the job to 2 minutes
    \end{itemize}
  \item Submit the job to the SLURM queue!
  \item Monitor the queue to see if your job is running
  \item What output did you get?
\end{itemize}
\end{frame}


%-------------------------------------------------------------------------------------
\begin{frame}[fragile]
  \frametitle{Submitting jobs}
\begin{exampleblock}{{Answer}}
    \verbatimfont{\footnotesize}
    \begin{itemize}
    \item The allocation, number of nodes and time are set with these flags
    \begin{verbatim}
      #SBATCH -A edu18.intropdc
      #SBATCH -t 0:02:00
      #SBATCH --nodes=2
    \end{verbatim}

    \item If you want to run 32 MPI processes in total, with 16 processes on each node, both the \verb|-n| and \verb|-N| flags 
      to \verb|aprun| must be used:
    \begin{verbatim}
      aprun -n 32 -N 16 ./hello_world_mpi
    \end{verbatim}

    \item The job is submitted and monitored like this:
    \begin{verbatim}
      $ sbatch sbatch_beskow.sh
      $ squeue -u <username>
    \end{verbatim}

    \item The output gets written to a default filename:
    \begin{verbatim}
      $ cat slurm-2559495.out
      Hello world from rank 0 out of 32 process
      Hello world from rank 4 out of 32 process
      ...
    \end{verbatim}

    \end{itemize}

\end{exampleblock}
\end{frame}


%-------------------------------------------------------------------------------------

\begin{frame}[fragile]
  \frametitle{SSH}
  \begin{alertblock}{SSH configuration (Linux and Mac)}
    \verbatimfont{\footnotesize}
    \begin{verbatim}
kthw@local~$ cat .ssh/config
# Hosts we want to authenticate to with Kerberos
Host *.kth.se *.kth.se.
# User authentication based on GSSAPI is allowed
GSSAPIAuthentication yes
# Key exchange based on GSSAPI may be used for server authentication
GSSAPIKeyExchange yes
# Hosts to which we want to delegate credentials
Host *.csc.kth.se *.csc.kth.se. *.nada.kth.se *.nada.kth.se. \
     *.pdc.kth.se *.pdc.kth.se.
# Forward (delegate) credentials (tickets) to the server.
GSSAPIDelegateCredentials yes
# Prefer GSSAPI key exchange
PreferredAuthentications gssapi-keyex,gssapi-with-mic
# All other hosts
Host *

 \end{verbatim}
\end{alertblock}

\end{frame}

%-------------------------------------------------------------------------------------

\begin{frame}[fragile]
  \frametitle{Kerberos}
  \begin{alertblock}{Kerberos configuration (Linux and Mac)}
    \verbatimfont{\footnotesize}
    \begin{verbatim}

kthw@local~$ cat /etc/krb5.conf
[domain_realm]
   .pdc.kth.se = NADA.KTH.SE
[appdefaults]
   forwardable = yes
   forward = yes
   krb4_get_tickets = no
[libdefaults]
   default_realm = NADA.KTH.SE
   dns_lookup_realm = true
   dns_lookup_kdc = true

 \end{verbatim}
\end{alertblock}

\end{frame}


%-------------------------------------------------------------------------------------

\begin{frame}[fragile]
  \frametitle{Kerberos}
  \begin{alertblock}{Create and list tickets}
    \verbatimfont{\footnotesize}
    \begin{verbatim}
kthw@local~$ klist
klist: No credentials cache found
      
kthw@local~$ kinit -f kthw@NADA.KTH.SE
Password for kthw@NADA.KTH.SE:

kthw@local~$ klist -Tf  
Ticket cache: KCM:501
Default principal: kthw@NADA.KTH.SE

Valid starting       Expires              Service principal
08/03/2017 16:39:56  08/04/2017 16:39:50  krbtgt/NADA.KTH.SE@NADA.KTH.SE
   Flags: FIA

  \end{verbatim}
  \end{alertblock}

\end{frame}


%-------------------------------------------------------------------------------------


\begin{frame}[fragile]
  \frametitle{Login}
  \begin{alertblock}{Log in to Beskow, check ticket}
    \verbatimfont{\footnotesize}
    \begin{verbatim}

kthw@local:~$ ssh kthw@beskow.pdc.kth.se
kthw@beskow-login2:~$ klist -f

Credentials cache: FILE:/tmp/krb5cc_H26527
    Principal: kthw@NADA.KTH.SE

Issued                Expires             Flags    Principal
Aug  3 16:41:51 2017  Aug  4 16:39:50 2017  FfA    krbtgt/NADA.KTH.SE@NADA.KTH.SE
Aug  3 16:41:52 2017  Aug  4 16:39:50 2017  fA     afs/pdc.kth.se@NADA.KTH.SE
Aug  3 16:41:52 2017  Aug  4 16:39:50 2017  fA     afs@NADA.KTH.SE

 \end{verbatim}
\end{alertblock}



\end{frame}


%-------------------------------------------------------------------------------------


\begin{frame}[fragile]
  \frametitle{Modules}
  \begin{alertblock}{Inspect module system}
    \verbatimfont{\footnotesize}
    \begin{verbatim}

kthw@beskow-login2:~$ module list
...
kthw@beskow-login2:~$ module avail
...
kthw@beskow-login2:~$ module avail gcc
...
kthw@beskow-login2:~$ CC -V
Cray C++ : Version 8.3.4  Mon Aug 07, 2017  15:04:06
kthw@beskow-login2:~$ module swap PrgEnv-cray PrgEnv-gnu
kthw@beskow-login2:~$ CC --version
g++ (GCC) 4.9.1 20140716 (Cray Inc.)

 \end{verbatim}
\end{alertblock}



\end{frame}


%-------------------------------------------------------------------------------------

\begin{frame}[fragile]
  \frametitle{Interactive job on Beskow}
  \begin{alertblock}{Go to Klemming and start interactive session}
    \verbatimfont{\footnotesize}
    \begin{verbatim}

kthw@beskow-login2:~$ cd /cfs/klemming/nobackup/k/kthw/

# (command line shortened below here)
$ salloc -A edu17.intrompi -N 1 -t 0:10:0 --res=edu-Dec-11
salloc: Granted job allocation 1733496

$ hostname
beskow-login2.pdc.kth.se

$ aprun -n 1 hostname
nid01610

$ exit
salloc: Relinquishing job allocation 1733497
salloc: Job allocation 1733497 has been revoked.

 \end{verbatim}
\end{alertblock}



\end{frame}


%-------------------------------------------------------------------------------------

\begin{frame}[fragile]
  \frametitle{Interactive job on Beskow}
  \begin{alertblock}{We compile and run MPI and OpenMP codes}
    \verbatimfont{\footnotesize}
    \begin{verbatim}

kthw@beskow-login2:~$ mkdir -p /cfs/klemming/nobackup/k/kthw/mpi_course
kthw@beskow-login2:~$ cd /cfs/klemming/nobackup/k/kthw/mpi_course
$ cp ~kthw/Public/intro_mpi_dec2017/hello_world.f90 .
$ module swap PrgEnv-cray PrgEnv-gnu
$ salloc -A edu17.intrompi --res=edu-Dec-11 -N 1 -t 0:10:0
salloc: Granted job allocation 1733496
$ ftn -O2 hello_world.f90 -o hello.x
$ aprun -n 32 ./hello.x

$ cp ~kthw/Public/intro_mpi_dec2017/omp_hello.c .
$ cc -fopenmp omp_hello.c -o omp_hello.x
$ export OMP_NUM_THREADS=32
$ aprun -n 1 -d 32 ./omp_hello.x

 \end{verbatim}
\end{alertblock}



\end{frame}


%-------------------------------------------------------------------------------------


\begin{frame}[fragile]
  \frametitle{Batch job}
  \begin{alertblock}{Compile code and write batch script (Beskow)}
    \verbatimfont{\footnotesize}
    \begin{verbatim}

$ cp ~/Public/intro_mpi_dec2017/hello_world.f90 .
$ ftn -o hello_world.x hello_world.f90

$ cat <<EOF > submit.bash
#!/bin/bash -l

#SBATCH -A edu17.intrompi
#SBATCH --reservation=edu-Dec-11
#SBATCH -J myjob
#SBATCH -t 0:10:00
#SBATCH -N 1
#SBATCH -e error_file.e
#SBATCH -o output_file.o

aprun -n 32 ./hello_world.x > my_output_file 2>&1

EOF

 \end{verbatim}
\end{alertblock}


\end{frame}


%-------------------------------------------------------------------------------------

\begin{frame}[fragile]
  \frametitle{Batch job}
  \begin{alertblock}{Submit and monitor job}
    \verbatimfont{\footnotesize}
    \begin{verbatim}

$ sbatch submit.bash
$ squeue -u kthw
JOBID USER ACCOUNT NAME  ST REASON START_TIME TIME TIME_LEFT NODES CPUS
1735211 kthw pdc.sta myjob R None 2017-08-07T16:31:01 0:00 10:00 1 64

$ cat my_output_file
 Hello from rank           31  of           32
 Hello from rank           13  of           32
 Hello from rank           26  of           32
 Hello from rank           10  of           32
 Hello from rank           17  of           32
 Hello from rank           14  of           32
 Hello from rank            1  of           32
...


 \end{verbatim}
\end{alertblock}


\end{frame}


%-------------------------------------------------------------------------------------


\frame{\huge\centering Questions...?}
