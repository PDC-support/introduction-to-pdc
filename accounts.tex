\section{Accounts}

\subsection*{Access requirements}
\frame{
\frametitle{Access requirements}

\begin{description}
\item [User account] either SUPR or PDC
\item [Time allocation] set the access limits
\end{description}

\begin{alertblock}{Apply for SUPR account}
\begin{itemize}
  \item http://supr.snic.se
  \item SNIC database of persons, projects, project proposals and more
  \item Apply and link SUPR account to PDC
\end{itemize}
\end{alertblock}

\begin{alertblock}{Apply for PDC account}
\begin{itemize}
  \item http://www.pdc.kth.se/support/accounts/user
  \item Electronic copy of your passport
  \item Valid post address for password 
  \item Membership of specific time allocation 
\end{itemize}
\end{alertblock}

}


\subsection{Time allocations}
\frame{
\frametitle{Time Allocations}
\scriptsize
\begin{block}{\large{Small allocation}}
\begin{itemize}
  \item Applicant can be a PhD student or higher
  \item Evaluated on a technical level only
  \item Limits is usually $5K$ corehours
\end{itemize}
\end{block}

\begin{exampleblock}{\large{Medium allocation}}
\begin{itemize}
  \item Applicant must be a senior scientist in Swedish academia
  \item Evaluated on a technical level only
  \item On large clusters: $200K$ corehours
\end{itemize}
\end{exampleblock}

\begin{alertblock}{\large{Large allocation}}
\begin{itemize}
  \item Applicant must be a senior scientist in Swedish academia
  \item Need evidence of successful work at a medium level
  \item Evaluated on a technical and scientific level
  \item Proposal evaluated by SNAC twice a year
\end{itemize}
\end{alertblock}
}

\subsection*{Acknowledgement }
\frame
{
\frametitle{Using resources}

\begin{itemize}
\item All resources are free of charge for Swedish academia
\item Acknowledgement \alert{are} taken into consideration when applying
\item Please acknowledge SNIC/PDC when using these resources:
\end{itemize}
\begin{alertblock}{Acknowledge SNIC/PDC}
The computations/simulations/[SIMILAR] were performed on resources provided by the Swedish National Infrastructure for Computing (SNIC) at [CENTERNAME (CENTER-ACRONYM)]
\end{alertblock}
\begin{alertblock}{Acknowledge people}
NN at [CENTER-ACRONYME] is acknowledged for assistance concerning technical and implementation aspects [OR SIMILAR] in making the code run on the [OR SIMILAR] [CENTER-ACRONYM] resources.
\end{alertblock}
}



\subsection{Authentication}
\begin{frame}[fragile]
\frametitle{Authentication}
\framesubtitle{\alert{\textbf{Kerberos}} Authentication Protocol}
\footnotesize
\begin{exampleblock}{\large{Ticket}}
\begin{itemize}
  \item Proof of users identity
  \item Users use passwords to obtain tickets
  \item Tickets are cached on the user's computer for a specified duration
  \item Tickets \alert{should be created on your local computer}
  \item No passwords are required during the ticket's lifetime
\end{itemize}
\end{exampleblock}

\begin{exampleblock}{\large{Realm}}
Sets boundaries within which an authentication server has authority \verb|NADA.KTH.SE|
\end{exampleblock}

\begin{exampleblock}{\large{Principal}}
Refers to the entries in the authentication server database  \verb|username@NADA.KTH.SE|
\end{exampleblock}

\end{frame}

\subsection*{Kerberos commands}

\begin{frame}[fragile]
\frametitle{Kerberos commands}
\begin{description}
 \item [kinit] proves identity
 \item [klist] lists kerberos tickets
 \item [kdestroy] destroys ticket file
 \item [kpasswd] changes password
\end{description}
\scriptsize
\begin{block}{}
  \begin{verbatim}
    $ kinit --forwardable username@NADA.KTH.SE
    $ klist -Tf
    
    Credentials cache : FILE:/tmp/krb5cc_500
        Principal: username@NADA.KTH.SE
    Issued       Expires     Flags  Principal
    Mar 25 09:45 Mar 25 19:45 FI krbtgt/NADA.KTH.SE@NADA.KTH.SE
    Mar 25 09:45 Mar 25 19:45 FA afs/pdc.kth.se@NADA.KTH.SE
\end{verbatim}
\end{block}
\end{frame}


\begin{frame}[fragile]
\frametitle{Login using Kerberos tickets}

\begin{block}{Get a 7 days forwardable ticket on your local system}
\begin{verbatim}
$ kinit -f -l 7d username@NADA.KTH.SE
\end{verbatim}
\end{block}

\begin{block}{Forward your ticket via ssh and login}
\begin{verbatim}
$ ssh
  -o GSSAPIDelegateCredential 
  -o GSSAPIAuthentication 
  -o GSSAPIKeyExchange 
  username@clustername.pdc.kth.se
\end{verbatim}
\end{block}

\begin{block}{OR, when using \url{~/.ssh/config}}
\begin{verbatim}
$ ssh username@clustername.pdc.kth.se
\end{verbatim}
\end{block}

\alert{Always create a kerberos ticket on your local system}
\end{frame}
